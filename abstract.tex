\begin{abstract}

While it is hard to dispute that the ICS department at UH Manoa is a diverse, supportive, and overall well-functioning community, over the last decade, recent alumni and current undergraduates have expressed several problems with various academic, professional, and social aspects of their ICS experience. Existing degree planning systems such as STAR, Starfish by Hobsons and Blackboard Planner fail to provide the specific support that an ICS student needs. Existing social networks such as LinkedIn and TechHui fail to connect students closely with professors and alumni. Current popular video games suggest several gamification features that could encourage ICS students to achieve higher goals at a healthy rate. A new system called RadGrad combines degree planning, social networking, and gamification in a specific way that caters specially towards undergraduate ICS students and gives them the support they need to succeed. By conducting baseline and post-RadGrad deployment questionnaires and interviews, by May 2017, we will discover whether or not RadGrad can be used as a tool to fix many of the ICS students' existing problems. 
\end{abstract}
