%%%%%%%%%%%%%%%%%%%%%%%%%%%%%% -*- Mode: Latex -*- %%%%%%%%%%%%%%%%%%%%%%%%%%%%
%% uhtest-abstract.tex -- 
%% Author          : Robert Brewer
%% Created On      : Fri Oct  2 16:30:18 1998
%% Last Modified By: Robert Brewer
%% Last Modified On: Fri Oct  2 16:30:25 1998
%% RCS: $Id: uhtest-abstract.tex,v 1.1 1998/10/06 02:06:30 rbrewer Exp $
%%%%%%%%%%%%%%%%%%%%%%%%%%%%%%%%%%%%%%%%%%%%%%%%%%%%%%%%%%%%%%%%%%%%%%%%%%%%%%%
%%   Copyright (C) 1998 Robert Brewer
%%%%%%%%%%%%%%%%%%%%%%%%%%%%%%%%%%%%%%%%%%%%%%%%%%%%%%%%%%%%%%%%%%%%%%%%%%%%%%%
%% 

\begin{abstract}
Static Observation Networks (SONs) are often used in the biological sciences to study animal migration and habitat.  These networks are comprised of self-contained, stationary receivers that continuously listen for acoustic transmissions released by sonic tags carried by individual animals.  The transmissions released by these tags carry serial identification numbers that can be used to verify that a particular individual was near a given receiver.  Because receivers in these networks are stationary, receiver placement is critical to maximizing data recovery.  Currently, no open-source automated mechanism exists to facilitate the design of optimal receiver networks.  SON design is often governed by loose ''rules of thumb'' and ''by eye'' readings of low resolution bathymetric maps.  Moreover, there is no standardized method for evaluating the efficacy of a SON.  This paper  introduces the Maximal Acoustic Network Designer (MANDe) a system which takes advantage of high-resolution bathymetric data and advanced animal modeling to provide optimal network designs.   MANDe also allows for statistical analysis of existing network configurations in order to create efficacy-metrics that can be used to evaluate arbitrary network configurations.  This paper will present MANDe's mathematical and conceptual models  and analyze the computational complexities of its methods.
\end{abstract}
