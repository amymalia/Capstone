\chapter{Introduction}
There are a lot of things that can make a department like no other--exceptional people, exceptional facilities, exceptional events, or exceptional work. I may be biased, but I believe the ICS department at UH Manoa is a department like no other--and is exceptional in all these ways. The people are exceptional--I came into this major not knowing what to expect, but the people I have met along the way have exceeded all of my expectations. I have met some of the most brilliant, the most passionate, the most interesting and the most genuine people in this department. The facilities are exceptional--the computer labs are well maintained, and the new ICSpace is a great place for members of the department to socialize and create a sense of community. The events are exceptional--within this major, there are so many opportunities to participate in events. There are workshops, hackathons, competitions, and open source projects just waiting for participants. Finally, the work itself is exceptional--again, I may be biased, but I have grown to become very passionate about computer science, and I have met many others in the department who feel the same way. Computer science is constantly changing and it is a challenge to keep up with it, but it definitely keeps things interesting.

But nothing is perfect. Data gathered from 199 ICS students from 2008 to 2016 on the Hawaii technology community site, TechHui, suggests that the following ten categories have constantly displeased students over the past 8 years:

\begin{enumerate}
  \item The ICS department needs to offer classes more frequently.
  \item The ICS department needs to offer a wider variety of classes.
  \item The ICS department needs a better sense of community.
  \item Some of the professors in the ICS department need to improve their teaching.
  \item The ICS department should offer more focused areas of study.
  \item ICS classes are too time consuming and take up more time than anticipated.
  \item The ICS department should offer more classes that meet focus requirements.
  \item ICS books are too expensive.
  \item Classes should be offered more frequently.
  \item ICS courses should involve more group work 
  \item ICS should encourage more interaction among students.
\end{enumerate}

Categories 1, 2, 5, 6, 7, 8, and 9 suggest problems with the coursework itself and categories 3, 4, and 10 suggest social and communication related problems within the department. There were also some other complaints among students on TechHui that were not as common but stuck out to me nonetheless. There were at least eight students who mentioned that they felt intimidated when they started out in ICS, due to the impressions they got from their classmates and the major overall. This discouraged them in several ways and had an overall negative impact on their ICS experience. These sentiments further suggest social problems with the ICS community, as well as with how the department is perceived outside of the community. Additionally, apart from the sentiments expressed on TechHui, several ICS alumni that I remain in contact with are currently having problems finding ideal jobs after graduation. Feedback from employers suggest that this trend may be due in part to the small computer science market in Hawaii and in part to the lack of professional experience of many students straight out of college. Statistics like this suggest existing problems with professional development within the ICS department. 

As ideal as it would be, it is hard to meet the needs of all current, past and present students in a department. However, after taking student and alumni feedback into consideration, several of these problems could potentially be alleviated by creating an online platform (what could computer science students want more?) that provides students with the help they need--academically, professionally, and socially. By combining three aspects (degree planner, social network, and gamification), a new system called RadGrad could address many of the aforementioned student problems and needs.  

