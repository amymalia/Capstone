\chapter{Conclusions}
MANDe is a framework for the optimization and measurement of Static Acoustic Observation Networks (SAONs), the benefits of which are decreased cost of data collection, increased volume of collected data, and higher data quality.  While these benefits are largely realized during the design of a SAON, the framework's utility extends to the augmentation and assessment of acoustic tracking studies.  The ability to include an already existing SAON in the design process allows for the optimal augmentation and integration of SAONs.  This is useful to projects that operate on annual budgetary allotments and periodically purchase and deploy of hardware.  By supporting this incremental growth pattern, MANDe can be used to make optimal use of limited resources.  This functionality also provides a mechanism for researchers from separate studies to pool resources, mesh disjoint networks, and collaboratively increase data recovery rates.  MANDe also provides automated metrics for the evaluation and comparison of SAONs.  These metrics facilitate the evaluation of existing and theoretical network topologies, which is useful when demonstrating the need for additional hardware.

Finally, MANDe is customizable, allowing users to develop their own Behavior, Evaluation, and Suppression Algorithms, Shape/Attenuation Functions, and metrics.  Coupled with the fact that the framework is open source (the source code is freely available and editable \cite{acousitcdeploy}), and free to use, it is expected that this tool will have a strong impact on the acoustic tracking community.  
